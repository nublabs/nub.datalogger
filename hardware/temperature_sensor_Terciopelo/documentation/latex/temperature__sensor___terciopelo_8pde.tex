\hypertarget{temperature__sensor___terciopelo_8pde}{
\section{/Users/nikki/nublabs/nub.datalogger/hardware/temperature\_\-sensor\_\-Terciopelo/temperature\_\-sensor\_\-Terciopelo.pde File Reference}
\label{temperature__sensor___terciopelo_8pde}\index{/Users/nikki/nublabs/nub.datalogger/hardware/temperature\_\-sensor\_\-Terciopelo/temperature\_\-sensor\_\-Terciopelo.pde@{/Users/nikki/nublabs/nub.datalogger/hardware/temperature\_\-sensor\_\-Terciopelo/temperature\_\-sensor\_\-Terciopelo.pde}}
}
{\tt \#include \char`\"{}name.h\char`\"{}}\par
{\tt \#include \char`\"{}globals.h\char`\"{}}\par
{\tt \#include \char`\"{}communications\_\-definitions.h\char`\"{}}\par
{\tt \#include $<$stdio.h$>$}\par


Include dependency graph for temperature\_\-sensor\_\-Terciopelo.pde:\subsection*{Defines}
\begin{CompactItemize}
\item 
\#define \hyperlink{temperature__sensor___terciopelo_8pde_658c2878485cfe5cc625d283a6d34bc1}{XBEE\_\-SLEEP}~2
\item 
\#define \hyperlink{temperature__sensor___terciopelo_8pde_b2de299215608c2a35f0feb86adc2f6f}{SAMPLE\_\-BUTTON}~3
\item 
\#define \hyperlink{temperature__sensor___terciopelo_8pde_eb7a7ba1ab7e0406f1b5ab36d579f585}{LED}~4
\item 
\#define \hyperlink{temperature__sensor___terciopelo_8pde_2a2946288d28852ba343b09fd4f17d7a}{SENSOR1\_\-TOP}~0
\item 
\#define \hyperlink{temperature__sensor___terciopelo_8pde_0da2a51dcb3e00b10aedd07d75f22382}{SENSOR1\_\-BOTTOM}~1
\item 
\#define \hyperlink{temperature__sensor___terciopelo_8pde_645141ae2ab7fa7ac3f690c4959b6baf}{SENSOR2\_\-TOP}~2
\item 
\#define \hyperlink{temperature__sensor___terciopelo_8pde_df66e6da6cf8c78004dc0a75fd14d3b1}{SENSOR2\_\-BOTTOM}~3
\end{CompactItemize}
\subsection*{Functions}
\begin{CompactItemize}
\item 
void \hyperlink{temperature__sensor___terciopelo_8pde_4fc01d736fe50cf5b977f755b675f11d}{setup} ()
\item 
void \hyperlink{temperature__sensor___terciopelo_8pde_fe461d27b9c48d5921c00d521181f12f}{loop} ()
\item 
void \hyperlink{temperature__sensor___terciopelo_8pde_50a2ce599e896bfb535e70a42003ed23}{sample} ()
\item 
int \hyperlink{temperature__sensor___terciopelo_8pde_f8c68e93feeba5b9244094043672bac0}{getByte} (int timeout)
\item 
int \hyperlink{temperature__sensor___terciopelo_8pde_8f2521044963073c55b3c290fffd79e3}{getMessage} (int timeout)
\item 
void \hyperlink{temperature__sensor___terciopelo_8pde_95b1b253ee46df6a93285803cf1f3370}{sendData} ()
\item 
unsigned char \hyperlink{temperature__sensor___terciopelo_8pde_465a79dc430d1e52a5b540920da744ca}{getChecksum} ()
\item 
void \hyperlink{temperature__sensor___terciopelo_8pde_e369b3765489ee8bd0ea791c1843630f}{configure} ()
\begin{CompactList}\small\item\em \hyperlink{nublogger_8h_e369b3765489ee8bd0ea791c1843630f}{configure()} runs if the computer is trying to change the sensor's sample rate \item\end{CompactList}\item 
void \hyperlink{temperature__sensor___terciopelo_8pde_3fdb2350c3f98c0de0f0ae3c831a8b14}{discover} ()
\begin{CompactList}\small\item\em This function tells the computer of the datalogger's existence. \item\end{CompactList}\item 
void \hyperlink{temperature__sensor___terciopelo_8pde_f6c9587ccbcf223f8c79f508c2fef366}{initializeSensor} ()
\begin{CompactList}\small\item\em this function configures all the digital communication pins as input or output pins \item\end{CompactList}\item 
void \hyperlink{temperature__sensor___terciopelo_8pde_ea28af0c7128421a38589128bb39ef1c}{getTemperatures} ()
\item 
void \hyperlink{temperature__sensor___terciopelo_8pde_cfc975251dbc3a8c9a9b11f8df62cc41}{getRawData} ()
\begin{CompactList}\small\item\em this just grabs the raw values off the analog to digital converter \item\end{CompactList}\item 
void \hyperlink{temperature__sensor___terciopelo_8pde_8e666a34a083b1806167ca991be0c436}{convertToResistance} ()
\begin{CompactList}\small\item\em this function converts the raw ADC values from the thermistor into resistances of the thermistor \item\end{CompactList}\item 
void \hyperlink{temperature__sensor___terciopelo_8pde_3aa4f99331713009a70ee34eba83754b}{convertToTemperature} ()
\end{CompactItemize}
\subsection*{Variables}
\begin{CompactItemize}
\item 
float \hyperlink{temperature__sensor___terciopelo_8pde_735577560ca40e5b6008a98829068904}{R0} = 10000.0
\item 
float \hyperlink{temperature__sensor___terciopelo_8pde_8188fea1f6709096fe21a3ee084d00d0}{B} = 3950.0
\item 
float \hyperlink{temperature__sensor___terciopelo_8pde_4211ba1269f650e21964d32238a460b2}{T0} = 298
\item 
float \hyperlink{temperature__sensor___terciopelo_8pde_d17df5990b551ac9e97a3d60f65833ff}{RBOTTOM} = 1000.0
\end{CompactItemize}


\subsection{Define Documentation}
\hypertarget{temperature__sensor___terciopelo_8pde_eb7a7ba1ab7e0406f1b5ab36d579f585}{
\index{temperature\_\-sensor\_\-Terciopelo.pde@{temperature\_\-sensor\_\-Terciopelo.pde}!LED@{LED}}
\index{LED@{LED}!temperature_sensor_Terciopelo.pde@{temperature\_\-sensor\_\-Terciopelo.pde}}
\subsubsection[{LED}]{\setlength{\rightskip}{0pt plus 5cm}\#define LED~4}}
\label{temperature__sensor___terciopelo_8pde_eb7a7ba1ab7e0406f1b5ab36d579f585}




Definition at line 293 of file temperature\_\-sensor\_\-Terciopelo.pde.\hypertarget{temperature__sensor___terciopelo_8pde_b2de299215608c2a35f0feb86adc2f6f}{
\index{temperature\_\-sensor\_\-Terciopelo.pde@{temperature\_\-sensor\_\-Terciopelo.pde}!SAMPLE\_\-BUTTON@{SAMPLE\_\-BUTTON}}
\index{SAMPLE\_\-BUTTON@{SAMPLE\_\-BUTTON}!temperature_sensor_Terciopelo.pde@{temperature\_\-sensor\_\-Terciopelo.pde}}
\subsubsection[{SAMPLE\_\-BUTTON}]{\setlength{\rightskip}{0pt plus 5cm}\#define SAMPLE\_\-BUTTON~3}}
\label{temperature__sensor___terciopelo_8pde_b2de299215608c2a35f0feb86adc2f6f}




Definition at line 292 of file temperature\_\-sensor\_\-Terciopelo.pde.\hypertarget{temperature__sensor___terciopelo_8pde_0da2a51dcb3e00b10aedd07d75f22382}{
\index{temperature\_\-sensor\_\-Terciopelo.pde@{temperature\_\-sensor\_\-Terciopelo.pde}!SENSOR1\_\-BOTTOM@{SENSOR1\_\-BOTTOM}}
\index{SENSOR1\_\-BOTTOM@{SENSOR1\_\-BOTTOM}!temperature_sensor_Terciopelo.pde@{temperature\_\-sensor\_\-Terciopelo.pde}}
\subsubsection[{SENSOR1\_\-BOTTOM}]{\setlength{\rightskip}{0pt plus 5cm}\#define SENSOR1\_\-BOTTOM~1}}
\label{temperature__sensor___terciopelo_8pde_0da2a51dcb3e00b10aedd07d75f22382}




Definition at line 296 of file temperature\_\-sensor\_\-Terciopelo.pde.\hypertarget{temperature__sensor___terciopelo_8pde_2a2946288d28852ba343b09fd4f17d7a}{
\index{temperature\_\-sensor\_\-Terciopelo.pde@{temperature\_\-sensor\_\-Terciopelo.pde}!SENSOR1\_\-TOP@{SENSOR1\_\-TOP}}
\index{SENSOR1\_\-TOP@{SENSOR1\_\-TOP}!temperature_sensor_Terciopelo.pde@{temperature\_\-sensor\_\-Terciopelo.pde}}
\subsubsection[{SENSOR1\_\-TOP}]{\setlength{\rightskip}{0pt plus 5cm}\#define SENSOR1\_\-TOP~0}}
\label{temperature__sensor___terciopelo_8pde_2a2946288d28852ba343b09fd4f17d7a}




Definition at line 295 of file temperature\_\-sensor\_\-Terciopelo.pde.\hypertarget{temperature__sensor___terciopelo_8pde_df66e6da6cf8c78004dc0a75fd14d3b1}{
\index{temperature\_\-sensor\_\-Terciopelo.pde@{temperature\_\-sensor\_\-Terciopelo.pde}!SENSOR2\_\-BOTTOM@{SENSOR2\_\-BOTTOM}}
\index{SENSOR2\_\-BOTTOM@{SENSOR2\_\-BOTTOM}!temperature_sensor_Terciopelo.pde@{temperature\_\-sensor\_\-Terciopelo.pde}}
\subsubsection[{SENSOR2\_\-BOTTOM}]{\setlength{\rightskip}{0pt plus 5cm}\#define SENSOR2\_\-BOTTOM~3}}
\label{temperature__sensor___terciopelo_8pde_df66e6da6cf8c78004dc0a75fd14d3b1}




Definition at line 298 of file temperature\_\-sensor\_\-Terciopelo.pde.\hypertarget{temperature__sensor___terciopelo_8pde_645141ae2ab7fa7ac3f690c4959b6baf}{
\index{temperature\_\-sensor\_\-Terciopelo.pde@{temperature\_\-sensor\_\-Terciopelo.pde}!SENSOR2\_\-TOP@{SENSOR2\_\-TOP}}
\index{SENSOR2\_\-TOP@{SENSOR2\_\-TOP}!temperature_sensor_Terciopelo.pde@{temperature\_\-sensor\_\-Terciopelo.pde}}
\subsubsection[{SENSOR2\_\-TOP}]{\setlength{\rightskip}{0pt plus 5cm}\#define SENSOR2\_\-TOP~2}}
\label{temperature__sensor___terciopelo_8pde_645141ae2ab7fa7ac3f690c4959b6baf}




Definition at line 297 of file temperature\_\-sensor\_\-Terciopelo.pde.\hypertarget{temperature__sensor___terciopelo_8pde_658c2878485cfe5cc625d283a6d34bc1}{
\index{temperature\_\-sensor\_\-Terciopelo.pde@{temperature\_\-sensor\_\-Terciopelo.pde}!XBEE\_\-SLEEP@{XBEE\_\-SLEEP}}
\index{XBEE\_\-SLEEP@{XBEE\_\-SLEEP}!temperature_sensor_Terciopelo.pde@{temperature\_\-sensor\_\-Terciopelo.pde}}
\subsubsection[{XBEE\_\-SLEEP}]{\setlength{\rightskip}{0pt plus 5cm}\#define XBEE\_\-SLEEP~2}}
\label{temperature__sensor___terciopelo_8pde_658c2878485cfe5cc625d283a6d34bc1}


these are the pin definitions for the v2 board

function atmega pin arduino pin

Serial RX PD0 0 //serial lines that go out to the xbee module Serial TX PD1 1 Xbee\_\-sleep PD2 2 //a pin that, when asserted, puts the xbee radio into a low power sleep mode Sample\_\-button PD3 3 //an optional button that forces the sensor to take and send out a measurement LED PD4 4 //an LED on board that you can use for all kinds of stuff

sensor1\_\-top PC0 (analog) 0 sensor1\_\-bottom PC1 (analog) 1 sensor2\_\-top PC2 (analog) 2 sensor2\_\-bottom PC3 (analog) 3 

Definition at line 291 of file temperature\_\-sensor\_\-Terciopelo.pde.

\subsection{Function Documentation}
\hypertarget{temperature__sensor___terciopelo_8pde_e369b3765489ee8bd0ea791c1843630f}{
\index{temperature\_\-sensor\_\-Terciopelo.pde@{temperature\_\-sensor\_\-Terciopelo.pde}!configure@{configure}}
\index{configure@{configure}!temperature_sensor_Terciopelo.pde@{temperature\_\-sensor\_\-Terciopelo.pde}}
\subsubsection[{configure}]{\setlength{\rightskip}{0pt plus 5cm}void configure ()}}
\label{temperature__sensor___terciopelo_8pde_e369b3765489ee8bd0ea791c1843630f}


\hyperlink{nublogger_8h_e369b3765489ee8bd0ea791c1843630f}{configure()} runs if the computer is trying to change the sensor's sample rate 

In \hyperlink{nublogger_8h_e369b3765489ee8bd0ea791c1843630f}{configure()}, the datalogger sends a LISTENING message to the computer, indicating that it's ready to receive data. The computer sends three ints: the hours, minutes and seconds of the sample interval, followed by a checksum byte that's the sum of the ints modulo 256. The sensor computes a checksum on the received data. If its checksum matches, it sends an ACKNOWLEDGE message back to the computer and updates its sample interval information. If the checksum does not match, it sends a CHECKSUM\_\-ERROR\_\-PLEASE\_\-RESEND message, asking the computer to send the three ints again, followed by a checksum. If the sensor can't get a valid message (with a matching checksum) after three tries, it gives up, sends a CHECKSUM\_\-ERROR\_\-GIVING\_\-UP message to the computer and keeps its original sample interval information 

Definition at line 162 of file temperature\_\-sensor\_\-Terciopelo.pde.

References buffer, CHECKSUM, CHECKSUM\_\-ERROR\_\-GIVING\_\-UP, CHECKSUM\_\-ERROR\_\-PLEASE\_\-RESEND, CONFIGURATION\_\-MESSAGE\_\-LENGTH, configured, getMessage(), HOUR\_\-HIGH, HOUR\_\-LOW, hours, index, LISTENING, MALFORMED\_\-MESSAGE\_\-ERROR\_\-GIVING\_\-UP, MALFORMED\_\-MESSAGE\_\-ERROR\_\-PLEASE\_\-RESEND, MESSAGE\_\-START, MINUTE\_\-HIGH, MINUTE\_\-LOW, minutes, NUM\_\-TRIES, SECOND\_\-HIGH, SECOND\_\-LOW, seconds, start, and TIMEOUT\_\-ERROR.

\begin{Code}\begin{verbatim}163 { 
164   char i=0;
165   char tries=0;
166   char success=0;
167   unsigned char checksum=0;
168   int error;
169   
170   while((tries<NUM_TRIES)&&(success==0))     //we'll try
171   {
172   checksum=0;
173   Serial.print(LISTENING,BYTE);
174   error=getMessage(100);
175   if(error==-1)
176     Serial.print(TIMEOUT_ERROR,BYTE);
177   else
178   {
179     if(buffer[start]==MESSAGE_START)
180     {
181       if((index-start)==CONFIGURATION_MESSAGE_LENGTH)    //check to make sure the message is the size we expect
182       {
183         for(i=1;i<CHECKSUM;i++)
184           checksum+=buffer[start+i];
185         if(checksum==buffer[start+CHECKSUM])    //check to see if the calculated checksum is the same as the received checksum
186         {
187           //if it is, then we can load all the sample interval info
188           hours=(int)buffer[start+HOUR_HIGH]*256+(int)buffer[start+HOUR_LOW];
189           minutes=(int)buffer[start+MINUTE_HIGH]*256+(int)buffer[start+MINUTE_LOW];
190           seconds=(int)buffer[start+SECOND_HIGH]*256+(int)buffer[start+SECOND_LOW];
191           success=1;         //we can stop looping
192           configured=1;      //the sensor is configured!
193         }
194         else
195         {
196           if(tries<NUM_TRIES)
197           {
198             Serial.print(CHECKSUM_ERROR_PLEASE_RESEND);
199             tries++;
200           }
201           else
202             Serial.print(CHECKSUM_ERROR_GIVING_UP);
203         }
204       }
205       else      //the message is the wrong size
206         {
207           if(tries<NUM_TRIES)
208           {
209             Serial.print(MALFORMED_MESSAGE_ERROR_PLEASE_RESEND);
210             tries++;
211           }
212           else
213             Serial.print(MALFORMED_MESSAGE_ERROR_GIVING_UP);
214         }
215     }
216     else
217         {
218           if(tries<NUM_TRIES)
219           {
220             Serial.print(MALFORMED_MESSAGE_ERROR_PLEASE_RESEND);
221             tries++;
222           }
223           else
224             Serial.print(MALFORMED_MESSAGE_ERROR_GIVING_UP);
225         }
226   }
227   }
228 }
\end{verbatim}
\end{Code}


\hypertarget{temperature__sensor___terciopelo_8pde_8e666a34a083b1806167ca991be0c436}{
\index{temperature\_\-sensor\_\-Terciopelo.pde@{temperature\_\-sensor\_\-Terciopelo.pde}!convertToResistance@{convertToResistance}}
\index{convertToResistance@{convertToResistance}!temperature_sensor_Terciopelo.pde@{temperature\_\-sensor\_\-Terciopelo.pde}}
\subsubsection[{convertToResistance}]{\setlength{\rightskip}{0pt plus 5cm}void convertToResistance ()}}
\label{temperature__sensor___terciopelo_8pde_8e666a34a083b1806167ca991be0c436}


this function converts the raw ADC values from the thermistor into resistances of the thermistor 



Definition at line 340 of file temperature\_\-sensor\_\-Terciopelo.pde.

References RBOTTOM, sensor1\_\-bottom, sensor1\_\-resistance, and sensor1\_\-top.

Referenced by getTemperatures().

\begin{Code}\begin{verbatim}341 {
342     sensor1_resistance = ((float)sensor1_top/(float)sensor1_bottom - 1)*RBOTTOM; // Voltages converted to resistances
343     /* uncomment if I enable two sensing elements
344     sensor2_resistance = ((float)sensor2_top/(float)sensor2_bottom - 1)*RBOTTOM; // Voltages converted to resistances*/
345  
346 }
\end{verbatim}
\end{Code}


\hypertarget{temperature__sensor___terciopelo_8pde_3aa4f99331713009a70ee34eba83754b}{
\index{temperature\_\-sensor\_\-Terciopelo.pde@{temperature\_\-sensor\_\-Terciopelo.pde}!convertToTemperature@{convertToTemperature}}
\index{convertToTemperature@{convertToTemperature}!temperature_sensor_Terciopelo.pde@{temperature\_\-sensor\_\-Terciopelo.pde}}
\subsubsection[{convertToTemperature}]{\setlength{\rightskip}{0pt plus 5cm}void convertToTemperature ()}}
\label{temperature__sensor___terciopelo_8pde_3aa4f99331713009a70ee34eba83754b}




Definition at line 349 of file temperature\_\-sensor\_\-Terciopelo.pde.

References B, R0, sensor1\_\-resistance, sensor1\_\-temperature, and T0.

Referenced by getTemperatures(), and sample().

\begin{Code}\begin{verbatim}350 {
351    sensor1_temperature = float(B/log(sensor1_resistance/(R0*exp(-1.0*B/T0))) - 273.0); // Temperature in degrees Celsius
352    /* uncomment if I enable two sensing elements
353    sensor2_temperature = float(B/log(sensor2_resistance/(R0*exp(-1.0*B/T0))) - 273.0); // Temperature in degrees Celsius   */
354 }
\end{verbatim}
\end{Code}


\hypertarget{temperature__sensor___terciopelo_8pde_3fdb2350c3f98c0de0f0ae3c831a8b14}{
\index{temperature\_\-sensor\_\-Terciopelo.pde@{temperature\_\-sensor\_\-Terciopelo.pde}!discover@{discover}}
\index{discover@{discover}!temperature_sensor_Terciopelo.pde@{temperature\_\-sensor\_\-Terciopelo.pde}}
\subsubsection[{discover}]{\setlength{\rightskip}{0pt plus 5cm}void discover ()}}
\label{temperature__sensor___terciopelo_8pde_3fdb2350c3f98c0de0f0ae3c831a8b14}


This function tells the computer of the datalogger's existence. 

When the sensor turns on, it runs \hyperlink{nublogger_8h_3fdb2350c3f98c0de0f0ae3c831a8b14}{discover()}. It sends a MESSAGE\_\-START message, a DISCOVER\_\-ME message, and its name out to the computer and waits for acknowledgement. The computer can send back a plain \char`\"{}ACKNOWLEDGE\char`\"{} message, which means that the sensor should run using its default configuration values. The computer can also send back an \char`\"{}ACKNOWLEDGE\_\-AND\_\-CONFIGURE\char`\"{} message, which means that it has configuration data for the sensor. If the sensor gets this message, it'll run \hyperlink{nublogger_8h_e369b3765489ee8bd0ea791c1843630f}{configure()} to receive the data from the computer. 

Definition at line 239 of file temperature\_\-sensor\_\-Terciopelo.pde.

References ACKNOWLEDGE, ACKNOWLEDGE\_\-AND\_\-CONFIGURE, configure(), DISCOVER\_\-ME, discovered, getByte(), MESSAGE\_\-END, MESSAGE\_\-START, name, TIMEOUT\_\-ERROR, and TRUE.

\begin{Code}\begin{verbatim}240 { 
241   unsigned char checksum=0;
242   int i=0;
243   Serial.print(MESSAGE_START, BYTE);
244   Serial.print(DISCOVER_ME,BYTE);
245   Serial.print(name);
246   while(name[i]!=0)
247   {
248     checksum+=name[i];
249     i++;
250   }
251   checksum+=DISCOVER_ME;
252   Serial.print(checksum,BYTE);
253   Serial.print(MESSAGE_END,BYTE);
254 
255   int receivedByte=getByte(100);     //looks for a byte on the serial port with a 100ms timeout
256   if(receivedByte==ACKNOWLEDGE)
257     discovered=TRUE;
258   if(receivedByte==ACKNOWLEDGE_AND_CONFIGURE)
259     {
260       discovered=TRUE;
261       configure();
262     }
263   if(receivedByte==-1)
264   {
265     Serial.print(TIMEOUT_ERROR,BYTE);  //getByte didn't get a byte before the timeout
266   }  
267 }
\end{verbatim}
\end{Code}


\hypertarget{temperature__sensor___terciopelo_8pde_f8c68e93feeba5b9244094043672bac0}{
\index{temperature\_\-sensor\_\-Terciopelo.pde@{temperature\_\-sensor\_\-Terciopelo.pde}!getByte@{getByte}}
\index{getByte@{getByte}!temperature_sensor_Terciopelo.pde@{temperature\_\-sensor\_\-Terciopelo.pde}}
\subsubsection[{getByte}]{\setlength{\rightskip}{0pt plus 5cm}int getByte (int {\em timeout})}}
\label{temperature__sensor___terciopelo_8pde_f8c68e93feeba5b9244094043672bac0}




Definition at line 92 of file temperature\_\-sensor\_\-Terciopelo.pde.

\begin{Code}\begin{verbatim}93 {
94   int currentTime=millis();
95   int maxTime=currentTime+timeout;
96   while((Serial.available()==0)&&(millis()<(maxTime)))
97     {}
98   if(Serial.available()>0)        //did any data come in on the serial port?
99     return Serial.read();
100   else                             //we didn't get any data before the timeout
101     return -1;
102 }
\end{verbatim}
\end{Code}


\hypertarget{temperature__sensor___terciopelo_8pde_465a79dc430d1e52a5b540920da744ca}{
\index{temperature\_\-sensor\_\-Terciopelo.pde@{temperature\_\-sensor\_\-Terciopelo.pde}!getChecksum@{getChecksum}}
\index{getChecksum@{getChecksum}!temperature_sensor_Terciopelo.pde@{temperature\_\-sensor\_\-Terciopelo.pde}}
\subsubsection[{getChecksum}]{\setlength{\rightskip}{0pt plus 5cm}unsigned char getChecksum ()}}
\label{temperature__sensor___terciopelo_8pde_465a79dc430d1e52a5b540920da744ca}




Definition at line 138 of file temperature\_\-sensor\_\-Terciopelo.pde.

References message.

Referenced by sendData().

\begin{Code}\begin{verbatim}139 {
140   char i=0;
141   unsigned char checksum=0;
142   while(message[i]!=0)
143   {
144     checksum+=message[i];
145     i++;
146   }
147   return checksum;
148 }
\end{verbatim}
\end{Code}


\hypertarget{temperature__sensor___terciopelo_8pde_8f2521044963073c55b3c290fffd79e3}{
\index{temperature\_\-sensor\_\-Terciopelo.pde@{temperature\_\-sensor\_\-Terciopelo.pde}!getMessage@{getMessage}}
\index{getMessage@{getMessage}!temperature_sensor_Terciopelo.pde@{temperature\_\-sensor\_\-Terciopelo.pde}}
\subsubsection[{getMessage}]{\setlength{\rightskip}{0pt plus 5cm}int getMessage (int {\em timeout})}}
\label{temperature__sensor___terciopelo_8pde_8f2521044963073c55b3c290fffd79e3}




Definition at line 104 of file temperature\_\-sensor\_\-Terciopelo.pde.

References buffer, index, MESSAGE\_\-END, and start.

Referenced by configure().

\begin{Code}\begin{verbatim}105 {
106   int completeMessage=-1;   //a flag that lets us know if we got a full message
107   start=index;              //drop whatever other data is in our buffer--it'll probably just confuse the functions if we don't
108   int currentTime=millis();
109   int maxTime=currentTime+timeout;  
110   while((millis()<(maxTime))&&(buffer[index]!=MESSAGE_END))
111     {
112       if(Serial.available()>0)
113         {
114           buffer[index]=Serial.read();
115           if(buffer[index]==MESSAGE_END)   //we got a complete message
116             completeMessage=1;
117           index++;
118         }
119     }
120   if(completeMessage==-1)    //we never got a complete message
121     start=index;    //skip past whatever we got from the buffer
122   
123   return completeMessage;
124 }
\end{verbatim}
\end{Code}


\hypertarget{temperature__sensor___terciopelo_8pde_cfc975251dbc3a8c9a9b11f8df62cc41}{
\index{temperature\_\-sensor\_\-Terciopelo.pde@{temperature\_\-sensor\_\-Terciopelo.pde}!getRawData@{getRawData}}
\index{getRawData@{getRawData}!temperature_sensor_Terciopelo.pde@{temperature\_\-sensor\_\-Terciopelo.pde}}
\subsubsection[{getRawData}]{\setlength{\rightskip}{0pt plus 5cm}void getRawData ()}}
\label{temperature__sensor___terciopelo_8pde_cfc975251dbc3a8c9a9b11f8df62cc41}


this just grabs the raw values off the analog to digital converter 



Definition at line 329 of file temperature\_\-sensor\_\-Terciopelo.pde.

References sensor1\_\-bottom, and sensor1\_\-top.

Referenced by getTemperatures(), and sample().

\begin{Code}\begin{verbatim}330 {
331   sensor1_top=analogRead(0);
332   sensor1_bottom=analogRead(1);
333   
334 /*  uncomment if I enable 2-sensing elements per sensor
335   sensor2_top=analogRead(2);
336   sensor2_bottom=analogRead(3);*/
337 }
\end{verbatim}
\end{Code}


\hypertarget{temperature__sensor___terciopelo_8pde_ea28af0c7128421a38589128bb39ef1c}{
\index{temperature\_\-sensor\_\-Terciopelo.pde@{temperature\_\-sensor\_\-Terciopelo.pde}!getTemperatures@{getTemperatures}}
\index{getTemperatures@{getTemperatures}!temperature_sensor_Terciopelo.pde@{temperature\_\-sensor\_\-Terciopelo.pde}}
\subsubsection[{getTemperatures}]{\setlength{\rightskip}{0pt plus 5cm}void getTemperatures ()}}
\label{temperature__sensor___terciopelo_8pde_ea28af0c7128421a38589128bb39ef1c}




Definition at line 321 of file temperature\_\-sensor\_\-Terciopelo.pde.

References convertToResistance(), convertToTemperature(), and getRawData().

\begin{Code}\begin{verbatim}322 {
323   getRawData();
324   convertToResistance();
325   convertToTemperature();
326 }
\end{verbatim}
\end{Code}


\hypertarget{temperature__sensor___terciopelo_8pde_f6c9587ccbcf223f8c79f508c2fef366}{
\index{temperature\_\-sensor\_\-Terciopelo.pde@{temperature\_\-sensor\_\-Terciopelo.pde}!initializeSensor@{initializeSensor}}
\index{initializeSensor@{initializeSensor}!temperature_sensor_Terciopelo.pde@{temperature\_\-sensor\_\-Terciopelo.pde}}
\subsubsection[{initializeSensor}]{\setlength{\rightskip}{0pt plus 5cm}void initializeSensor ()}}
\label{temperature__sensor___terciopelo_8pde_f6c9587ccbcf223f8c79f508c2fef366}


this function configures all the digital communication pins as input or output pins 

If you adapt this code to work with another sensor or board, you should replace the code in \hyperlink{temperature__sensor__board__v2_8h_f6c9587ccbcf223f8c79f508c2fef366}{initializeSensor()} to initialize all your relevant pins 

Definition at line 313 of file temperature\_\-sensor\_\-Terciopelo.pde.

References LED, SAMPLE\_\-BUTTON, and XBEE\_\-SLEEP.

\begin{Code}\begin{verbatim}314  {
315    pinMode(XBEE_SLEEP,OUTPUT);
316    pinMode(SAMPLE_BUTTON,INPUT);
317    pinMode(LED,OUTPUT);
318  }  
\end{verbatim}
\end{Code}


\hypertarget{temperature__sensor___terciopelo_8pde_fe461d27b9c48d5921c00d521181f12f}{
\index{temperature\_\-sensor\_\-Terciopelo.pde@{temperature\_\-sensor\_\-Terciopelo.pde}!loop@{loop}}
\index{loop@{loop}!temperature_sensor_Terciopelo.pde@{temperature\_\-sensor\_\-Terciopelo.pde}}
\subsubsection[{loop}]{\setlength{\rightskip}{0pt plus 5cm}void loop ()}}
\label{temperature__sensor___terciopelo_8pde_fe461d27b9c48d5921c00d521181f12f}




Definition at line 79 of file temperature\_\-sensor\_\-Terciopelo.pde.

\begin{Code}\begin{verbatim}80 {
81 }
\end{verbatim}
\end{Code}


\hypertarget{temperature__sensor___terciopelo_8pde_50a2ce599e896bfb535e70a42003ed23}{
\index{temperature\_\-sensor\_\-Terciopelo.pde@{temperature\_\-sensor\_\-Terciopelo.pde}!sample@{sample}}
\index{sample@{sample}!temperature_sensor_Terciopelo.pde@{temperature\_\-sensor\_\-Terciopelo.pde}}
\subsubsection[{sample}]{\setlength{\rightskip}{0pt plus 5cm}void sample ()}}
\label{temperature__sensor___terciopelo_8pde_50a2ce599e896bfb535e70a42003ed23}




Definition at line 83 of file temperature\_\-sensor\_\-Terciopelo.pde.

References convertToTemperature(), getRawData(), and sendData().

\begin{Code}\begin{verbatim}84 {
85   getRawData();
86   convertToTemperature();
87   sendData();
88 }
\end{verbatim}
\end{Code}


\hypertarget{temperature__sensor___terciopelo_8pde_95b1b253ee46df6a93285803cf1f3370}{
\index{temperature\_\-sensor\_\-Terciopelo.pde@{temperature\_\-sensor\_\-Terciopelo.pde}!sendData@{sendData}}
\index{sendData@{sendData}!temperature_sensor_Terciopelo.pde@{temperature\_\-sensor\_\-Terciopelo.pde}}
\subsubsection[{sendData}]{\setlength{\rightskip}{0pt plus 5cm}void sendData ()}}
\label{temperature__sensor___terciopelo_8pde_95b1b253ee46df6a93285803cf1f3370}




Definition at line 127 of file temperature\_\-sensor\_\-Terciopelo.pde.

References getChecksum(), message, MESSAGE\_\-END, MESSAGE\_\-START, and sensor1\_\-temperature.

Referenced by sample().

\begin{Code}\begin{verbatim}128 {
129   sprintf(message, "thermistor 1 = %f degrees C", sensor1_temperature);
130   unsigned char checksum=getChecksum();
131   Serial.print(MESSAGE_START);
132   Serial.print(message);
133   Serial.print(checksum);
134   Serial.print(MESSAGE_END);
135 }
\end{verbatim}
\end{Code}


\hypertarget{temperature__sensor___terciopelo_8pde_4fc01d736fe50cf5b977f755b675f11d}{
\index{temperature\_\-sensor\_\-Terciopelo.pde@{temperature\_\-sensor\_\-Terciopelo.pde}!setup@{setup}}
\index{setup@{setup}!temperature_sensor_Terciopelo.pde@{temperature\_\-sensor\_\-Terciopelo.pde}}
\subsubsection[{setup}]{\setlength{\rightskip}{0pt plus 5cm}void setup ()}}
\label{temperature__sensor___terciopelo_8pde_4fc01d736fe50cf5b977f755b675f11d}


nublogger temperature sensor Terciopelo(+)

Alex Hornstein 11.19.08 \href{mailto:alex@nublabs.com}{\tt alex@nublabs.com} datalogger.nublabs.com

(+)In keeping with the arduino nomenclature, we are naming all our code revisions with Spanish names of venemous snakes. The Terciopelo, or fer-de-lance is a pit viper common in central and northwestern south america. It has a powerful venom that, left untreated, can cause necrosis, brain hemorrhaging, renal failure and death. It's also capable of spraying venom through its fangs for up to 6 feet. Cool, huh? This is a sensor for nublab's datalogging system. The datalogger is a two-part system: There is a USB dongle that plugs into a computer that allows the computer to talk to a wireless Zigbee network, and then there are battery-powered sensors that sense data about the environemnt. The sensors collect data and convert it into human-readable units, and then send the data as plaintext over the wireless network to the computer, where it is stored and logged. Any sensor that will work with this system must implement the \hyperlink{nublogger_8h_3fdb2350c3f98c0de0f0ae3c831a8b14}{discover()}, \hyperlink{nublogger_8h_e369b3765489ee8bd0ea791c1843630f}{configure()} and \hyperlink{temperature__sensor___terciopelo_8pde_50a2ce599e896bfb535e70a42003ed23}{sample()} functions, as well as be identifiable by a unique name

The \hyperlink{nublogger_8h_3fdb2350c3f98c0de0f0ae3c831a8b14}{discover()} function is a short communication sequence when the sensor is first turned on where it broadcasts its name over the network and ensures that the computer recognizes it and is ready to configure it and log its data. The sensor also sends the units of whatever value it will be reporting.

the \hyperlink{nublogger_8h_e369b3765489ee8bd0ea791c1843630f}{configure()} function is triggered by a flag sent by the computer that indicates that the computer would like to change the datalogger's sample rate. \hyperlink{nublogger_8h_e369b3765489ee8bd0ea791c1843630f}{configure()} is another communication sequence in which the computer sends a sample interval in hours, minutes and seconds to the datalogger. By default, the datalogger samples every second.

the \hyperlink{temperature__sensor___terciopelo_8pde_50a2ce599e896bfb535e70a42003ed23}{sample()} function is called by the sensor every sample interval. \hyperlink{temperature__sensor___terciopelo_8pde_50a2ce599e896bfb535e70a42003ed23}{sample()} reads a value from whatever sensing element (Thermistor, current sensor, light sensor, etc) the particular sensor uses, converts it to physical units (degrees celsius, amps, lux, etc) and sends out a string over the wireless network with the sensor's unique name and its sensed values.

the name: each sensor should have a unique name burnt into its eeprom that makes it uniquely identifiable in the network. Nublabs is using a list of north and south american baby names which is available at datalogger.nublabs.com This particular sensor is a temperature sensor using an NTC thermistor. The thermistor is a resistive element. We sense it using two resistor dividers--one resistor, R1 which connects from Vcc to the thermistor, and another resistor that connects from the other end of the thermistor to ground. We measure the voltage at both ends of the thermistor using separate Analog to Digital Converter (ADC) pins. This allows us to factor out any effect changing battery voltage has on our temperature measurement. This code is written for version 2 of the sensor hardware. A pdf of the sensor schematic and board layout is available at datalogger.nublabs.com 

Definition at line 72 of file temperature\_\-sensor\_\-Terciopelo.pde.

References discover(), and initializeSensor().

\begin{Code}\begin{verbatim}73 {
74   Serial.begin(19200);
75   initializeSensor();
76   discover();
77 }
\end{verbatim}
\end{Code}




\subsection{Variable Documentation}
\hypertarget{temperature__sensor___terciopelo_8pde_8188fea1f6709096fe21a3ee084d00d0}{
\index{temperature\_\-sensor\_\-Terciopelo.pde@{temperature\_\-sensor\_\-Terciopelo.pde}!B@{B}}
\index{B@{B}!temperature_sensor_Terciopelo.pde@{temperature\_\-sensor\_\-Terciopelo.pde}}
\subsubsection[{B}]{\setlength{\rightskip}{0pt plus 5cm}float {\bf B} = 3950.0}}
\label{temperature__sensor___terciopelo_8pde_8188fea1f6709096fe21a3ee084d00d0}




Definition at line 302 of file temperature\_\-sensor\_\-Terciopelo.pde.

Referenced by convertToTemperature().\hypertarget{temperature__sensor___terciopelo_8pde_735577560ca40e5b6008a98829068904}{
\index{temperature\_\-sensor\_\-Terciopelo.pde@{temperature\_\-sensor\_\-Terciopelo.pde}!R0@{R0}}
\index{R0@{R0}!temperature_sensor_Terciopelo.pde@{temperature\_\-sensor\_\-Terciopelo.pde}}
\subsubsection[{R0}]{\setlength{\rightskip}{0pt plus 5cm}float {\bf R0} = 10000.0}}
\label{temperature__sensor___terciopelo_8pde_735577560ca40e5b6008a98829068904}




Definition at line 301 of file temperature\_\-sensor\_\-Terciopelo.pde.

Referenced by convertToTemperature().\hypertarget{temperature__sensor___terciopelo_8pde_d17df5990b551ac9e97a3d60f65833ff}{
\index{temperature\_\-sensor\_\-Terciopelo.pde@{temperature\_\-sensor\_\-Terciopelo.pde}!RBOTTOM@{RBOTTOM}}
\index{RBOTTOM@{RBOTTOM}!temperature_sensor_Terciopelo.pde@{temperature\_\-sensor\_\-Terciopelo.pde}}
\subsubsection[{RBOTTOM}]{\setlength{\rightskip}{0pt plus 5cm}float {\bf RBOTTOM} = 1000.0}}
\label{temperature__sensor___terciopelo_8pde_d17df5990b551ac9e97a3d60f65833ff}




Definition at line 304 of file temperature\_\-sensor\_\-Terciopelo.pde.

Referenced by convertToResistance().\hypertarget{temperature__sensor___terciopelo_8pde_4211ba1269f650e21964d32238a460b2}{
\index{temperature\_\-sensor\_\-Terciopelo.pde@{temperature\_\-sensor\_\-Terciopelo.pde}!T0@{T0}}
\index{T0@{T0}!temperature_sensor_Terciopelo.pde@{temperature\_\-sensor\_\-Terciopelo.pde}}
\subsubsection[{T0}]{\setlength{\rightskip}{0pt plus 5cm}float {\bf T0} = 298}}
\label{temperature__sensor___terciopelo_8pde_4211ba1269f650e21964d32238a460b2}




Definition at line 303 of file temperature\_\-sensor\_\-Terciopelo.pde.

Referenced by convertToTemperature().